\documentclass{article}
\usepackage[UTF8]{ctex}
\usepackage{geometry}
\usepackage{multirow}
\usepackage{natbib}
\geometry{left=3.18cm,right=3.18cm,top=2.54cm,bottom=2.54cm}
\usepackage{graphicx}
\pagestyle{plain}	
\usepackage{setspace}
\usepackage{enumerate}
\usepackage{caption2}
\usepackage{datetime} %日期
\renewcommand{\today}{\number\year 年 \number\month 月 \number\day 日}
\renewcommand{\captionlabelfont}{\small}
\renewcommand{\captionfont}{\small}
\begin{document}

\begin{figure}
    \centering
    \includegraphics[width=8cm]{upc.png}

    \label{figupc}
\end{figure}

	\begin{center}
		\quad \\
		\quad \\
		\heiti \fontsize{45}{17} \quad \quad \quad 
		\vskip 1.5cm
		\heiti \zihao{2} 《计算科学导论》个人职业规划
	\end{center}
	\vskip 2.0cm
		
	\begin{quotation}
% 	\begin{center}
		\doublespacing
		
        \zihao{4}\par\setlength\parindent{7em}
		\quad 

		学生姓名:\underline{\qquad  汪子霄 \qquad \qquad}

		学\hspace{0.61cm} 号:\underline{\qquad 1907010222\qquad}
		
		专业班级:\underline{\qquad 计科1902 \qquad  }
		
        学\hspace{0.61cm} 院:\underline{计算机科学与技术学院}
% 	\end{center}
		\vskip 1.5cm
		\centering
		\begin{table}[h]
            \centering 
            \zihao{4}
            \begin{tabular}{|c|c|c|c|c|c|c|c|c|}
            % 这里的rl 与表格对应可以看到,姓名是r,右对齐的;学号是l,左对齐的;若想居中,使用c关键字。
                \hline
                \multicolumn{5}{|c|}{分项评价} &\multicolumn{2}{c|}{整体评价}  & 总    分 & 评 阅 教 师\\
                \hline
                自我 & 环境 & 职业 & 实施 & 评估与 & 完整性 & 可行性 &\multirow{2}*{} &\multirow{2}*{}\\
                分析& 分析& 定位 & 方案 & 调整 & 20\% & 20\% & ~&~ \\\            
                10\% & 10\% & 15\% & 15\% & 10\% & &  &~ &~\\
                \cline{1-7} 
                & & & & & & & ~&~ \\
                & & & & & & & ~&~ \\
                \hline      
            \end{tabular}
        \end{table}
		\vskip 2cm
		\today
	\end{quotation}

\thispagestyle{empty}
\newpage
\setcounter{page}{1}
% 在这之前是封面,在这之后是正文
\section{自我分析}
	自我分析,是指对自我理性、深刻、全面的分析,他是自我介绍与自我评价的结合。进行自我分析对每个人来说都是非常有必要的,人在不断的变化、进步,自我的分析也应该不断的更新。古人有曰:“知彼知己,百战不殆”,相比孙老师一定是为了让我们正确知己,向我们颁布这次规划。而作为新晋大学生,我非常感谢孙老师的这次指导,我也认为只有认识了自己,才能对自己的职业做出正确的选择,才能选定适合自己发展的职业生涯路线,才能对自己的职业生涯目标做出最佳抉择。\par

\subsection{自然条件}
我是一名十八岁的男大学生,健康状态保持在非常优秀的水平。我来自山东东营,出自孙武故里广饶的一个普通家庭,拥有着朴素的外貌与不平凡的内心。\par

\subsection{性格分析}
我性格乐观开朗,心里活动丰富;敢于探索有挑战的问题,能够沉着冷静地面对问题,能够在兴趣的驱动下探究问题的本质。我追求上进,能够在帮助他人的过程中发现个人的价值。\par

\subsection{教育与学习经历}
顺利接受九年义务教育,在高中接触信息学竞赛,初次接受到C++语言,虽然只是面向过程的简单问题的解决方法,但它打开了我新知识领域的大门,丰富了高中知识的学习。\par
继高中学习信息学竞赛之后,我在大学选择了计算机科学与技术专业。计算科学导论的学习,让我发现了计算科学的新天地,程序设计的课程,也让我发现了更多富有特点的编程语言(Java,Python)。大学尚未结束,我相信大一构建大体规划后,今后的学习将会更加深入人心。\par

\subsection{工作与社会阅历}
\begin{enumerate}[(1)]
	\item 为学弟学妹讲过C++基础算法课程
	\item 做过初高中衔接班数学助理老师
	\item 乐安公园环保志愿者
	\item 敬老院陪伴老人志愿者
\end{enumerate}
\par

\subsection{知识、技能与经验}
\begin{enumerate}[(1)]
	\item 熟练使用office
	\item 掌握高中的数学物理化学知识
	\item 掌握基本的面向过程的C++代码实现
\end{enumerate}
\par

\subsection{兴趣爱好与特长}
我喜欢听音乐,非常享受在音乐的还要中思考问题与近期的计划。长跑是我锻炼身体的主要方式,我喜欢在奔跑中忘却烦恼。阅读是我放松思想的主要方式,正如孙老师倡导,我时常翻阅《计算科学导论》,在其中发现了我大学的课程大致内容与目前主打方向——网络安全。\par

\section{环境分析}

\subsection{社会环境分析}
习主席深刻指出:“没有网络安全就没有国家安全,就没有经济社会稳定运行,广大人民群众利益也难以得到保障。”作为全球网民最多,联网区域最广,信息化程度与日俱增的网络大国,要确保网络安全,必须构筑好网络安全屏障,确保网络安全责任落到实处。应对网络和信息安全挑战,关键在人。网络空间安全、网络信息安全,在如今大数据运行的今天,我们应培养对自身信息安全基本的防范意识。\par
互联网的发展,种种网络病毒与网络犯罪也随之而来,为了减少和防止该类犯罪给企业和个人带来的隐患,网络安全工程师这一神秘职业逐渐为人们所熟悉。社会对信息安全服务的需求很大,军队、国防、银行、税务、证券、机关、电子商务都急需大批网络安全人才,网络安全工程师已跻身IT新贵之列。在我国,根据国家信息化建设的规模保守估计,2011年国内网络安全专业人才仍存在近百万的巨大缺口,高级的战略人才和专业技术人才尤其匮乏。\par
目前,中国网络安全人才正处在一个起步过程,网络上充斥着大量黑客培训,多属于技巧训练,和真正的网络安全人才培养有本质区别,是完全不同的概念。如果这种局面不进一步改变,网络安全人才的空档只会继续拉大,直接导致国家经济损失加大。\par
作为世界第二大经济体,我国经济仍处于相对高速发展过程中,当今世界正处于百年未有之大变局,经济与生活都在高速发展。\par
如今,5G问题备受政界、商界及学界的普遍关注和广泛讨论,并已成为中美高科技竞争的核心议题,而且5G技术对网络空间安全的影响也是不可忽视的。中国必将需要大量的网络安全工程师,更需要优秀的工程师。\par

\subsection{家庭环境分析}
目前未婚,经济条件一般,基本可以跟随国家的发展,成为小康家庭。我父母都对我充满信心,富有期望,他们希望我可以有自己的方向,有自己的规划,获得自己期望的生活,他们并不会向我过度施压,更多的是时刻鼓励我,为我排忧解难。我的家庭中每个人都秉承艰苦奋斗的精神。\par
\subsection{职业环境分析}
市场需求分析:当今社会更需要的的人才,而人才又分为全才和专才,中国政治稳定,经济持续发展。在全球经济一体化环境中的重要角色。经济发展有强劲的势头,加入WTO后,会有大批的外国企业进入中国市场,中国的企业也将走出国门。\par
职业观察分析:中国的计算机人才不强,而专业人士更少,并且中国科技还不够发达,所以在中国有更多的职业需求,更容易就业。\par
行业分析:计算机软件主流开发技术、软件工程、软件项目过程管理等基本知识与技能、熟练掌握先进的软件开发工具、环境和软件工程管理方法,培养学生系统的软件设计与项目实施能力,胜任软件开发、管理和维护等相关工作的专业性软件
工程高级应用型人才。\par


\subsection{地域与人际环境分析}
我理想的城市坐标南方,而深圳被誉为“中国硅谷”,我也将其作为我的奋斗目标。\par
深圳地处中国华南地区、广东南部、珠江口东岸,东临大亚湾和大鹏湾,西濒珠江口和伶仃洋,南隔深圳河与香港相连,所处纬度较低,属亚热带海洋性气候。由于深受季风的影响,夏季盛行偏东南风,时有季风低压、热带气旋光顾,高温多雨;其余季节盛行东北季风,天气较为干燥,气候温和。他是粤港澳大湾区四大中心城市之一 、国际性综合交通枢纽、国际科技产业创新中心 、中国三大全国性金融中心之一 ,并全力建设全球海洋中心城市、中国特色社会主义先行示范区、综合性国家科学中心。深圳水陆空铁口岸俱全,是中国拥有口岸数量最多、出入境人员最多、车流量最大的口岸城市。\par
深圳是中国经济中心城市之一,经济总量长期位列中国大陆城市第四位,是中国大陆经济效益最好的城市之一。英国《经济学人》2012年“全球最具经济竞争力城市”榜单上,深圳位居第二。中国优秀企业华为更是坐标深圳,城市的优秀决定人才的优秀,我相信与优秀的人竞争可以得到优秀的人际关系与提升程度。\par 

\section{职业定位}

\subsection{行业领域定位与理由}
从事网络信息安全领域。\par
网络工程是国家战略工程,网络安全问题关系到国家的安全与社会的稳定,在网络信息技术高速发展的今天,在全球化进程的不断加速中,网络安全的重要性被日益放大,解决存在的安全问题变得非常迫切。其重要性正随着全球信息化步伐的加快而变得越来越重要,安全问题需要迫切解决。从微观角度看,网络安全关系到一个单位的发展。在企事业单位中,无论是科研数据还是财务数据,甚至是人力资源数据,都需要通过计算机网络来处理,特别是自动化办公的推广,企事业单位的很多工作都已经离不开网络,但是网络又离不开安全支撑,一旦出现安全问题,其后果是非常严重的。在网络工程中,安全技术的运用占据着非常重要的地位。\par
目前,许多企事业单位的业务依赖于信息系统安全运行,信息安全重要性日益凸显。信息已经成为各企事业单位中重要资源,也是一种重要的“无形财富”,在未来竞争中谁获取信息优势,谁就掌握了竞争的主动权。信息安全已成为影响国家安全、经济发展、社会稳定、个人利害的重大关键问题。\par

\subsection{职业岗位起点定位与理由}
网络安全工程师\par
对网络安全的探索是我的兴趣驱动,可以让我的效率事半功倍。我认为以网络安全为起点,可以进一步夯实网络基础,为未来方向提供一定的实力基础。\par

\subsection{职业目标与可行性分析}
\begin{enumerate}[(1)]
	\item 短期目标(大学4年)
	\begin{table}[h]
		\centering 
		\zihao{5}
		\begin{tabular}{|c|c|c|}
			% 这里的rl 与表格对应可以看到,姓名是r,右对齐的;学号是l,左对齐的;若想居中,使用c关键字。
			\hline
			年级 & 主要方向 & 课外计划\\
			\hline
			大学 & 高等数学、离散数学、 & 学习Java、HTML、JavaScrip\\
			一年级 &  线性代数、C++、python & 巩固ACM知识,准备省赛 \\ 
			\hline
			大学 & 掌握专业课知识 & 考虑大学生创新与数学建模竞赛\\
			二年级 & 备考CET-4 &  加强ACM代码能力,争取区域赛\\
			\hline
			大学 & 学习信息安全等专业知识 & 学习JavaServer Pages、Linux \\
			三年级 & 备考CET-6 & 在继续ACM与其他项目中做出选择 \\
			\hline
			大学 & 在备考研究生与 & 研究大学实力或者 \\
			四年级 & 工作之间做出选择 & 了解企业的招聘信息 \\
			\hline
		\end{tabular}
	\end{table}
	\item 中长期目标(5-10年)\par
	保持自主学习的能力。\par
	研究生方向选择网络安全,发表自己对网络安全理解的论文 \par
	就业选择网络安全工程师,不断加强自己的能力,不局限与单一性学习。\par
\end{enumerate}


\section{实施方案}
\begin{enumerate}[1、]
	\item 充分利用乐于探索的精神,保持并发扬对数学的热爱。
	\item 克服拖延症,提升英语水平,利用琐碎时间阅读英文原著。
	\item 通过询问老师或网上资源的途径自主训练,目标能够担负起小型网络信息安全工作。对网络信息安全有较为完整的认识,掌握电脑安全防护、网站安全、电子邮件安全、Internet网络安全部署、操作系统安全配置、恶意代码防护、常用软件安全设置、防火墙的应用等技能。
	\item 在深入专业训练专业训练,能完善和优化企业信息安全制度和流程。信息安全工作符合特定的规范要求,能够对系统中安全措施的实施进行了跟踪和验证,能够建立起立体式、纵深的安全防护系统,部署安全监控机制,对未知的安全威胁能够进行预警和追踪。
	\item 利用音乐与跑步发泄情绪,善于发泄学会发泄
\end{enumerate}
\par 

\section{评估与调整}

\subsection{评估时间}
每学期评估一次。\par 

\subsection{评估内容}
计算机语言的掌握水平,竞赛成果,英语水平,课程学分绩。\par

\subsection{调整原则}
自身情况的匹配性、与环境的适应性、操作实施的可行性等。\par


\end{document}
